% Currently this document is written in English
% !TeX encoding = UTF-8
% !TeX spellcheck = en_US

%Ensure that all odl school LaTeX habits are remarked
\RequirePackage[l2tabu, orthodox]{nag}

%German: remove "english"
\documentclass[english,utf8,biblatex]{lni-tex/lni}
% Nice tables using \toprule, \midrule, \bottomrule
\usepackage{booktabs}
% subfigures and subcaptions
\usepackage{subcaption}

%% Begin: Drawings
% use standalone and tikz for high-fid. drawings

% standalone package and config
\usepackage{standalone} % For pre-compiled pictures
\standaloneconfig{mode=buildnew} % only build image if source file is newer

% tikz package and config
\usepackage{tikz} % For Tikz pictures
\usetikzlibrary{
  positioning,
  fit,
  arrows,
  calc,
  backgrounds
}
\usepackage{pgf-umlsd} % UML sequence diagrams
\usepackage[simplified,school]{pgf-umlcd} % UML class diagrams
% UML class diagram overrides
\renewcommand{\umldrawcolor}{black}
\renewcommand{\umlfillcolor}{white}
\tikzstyle{package}=[font=\bf]
% UML class diagram dactor env
\newenvironment{dactor}[1]{
  \def\umlcdDactorName{#1}
  \def\umlcdPackageFit{}
}{
  \begin{pgfonlayer}{background}
    \node[umlcd style, draw, inner sep=0.5cm, fit=\umlcdPackageFit] (\umlcdDactorName) {};
    \node[umlcd style, draw, outer ysep=-0.5, anchor=south west] (\umlcdDactorName caption) at
    (\umlcdDactorName.north west) {\bf\umlcdDactorName : Dactor};
  \end{pgfonlayer}
}
% pgfplots
\usepackage{pgfplots}
\pgfplotsset{compat=1.16} % set pgfplots compatibility to version 1.16
%% End: Drawings

%% Begin: Biblatex

%for easy quotations: \enquote{text}, also required by biblatex
\usepackage{csquotes}
% biblatex is included with LNI-class option: `biblatex`, only set bibliography-file:
\bibliography{paper}

% Clear fields we do not need
\iffalse
\AtEveryBibitem{%
  \ifentrytype{article}{%
  }{%
    \clearfield{doi}%
    \clearfield{issn}%
    \clearfield{url}%
    \clearfield{urldate}%
  }%
  \ifentrytype{inproceedings}{%
  }{%
    \clearfield{doi}%
    \clearfield{issn}%
    \clearfield{url}%
    \clearfield{urldate}%
  }%
}
\fi
%% End: Biblatex

%% Begin: lstlistings

% Scala highlighting
\lstdefinelanguage{scala}{
  morekeywords={%
    abstract,case,catch,class,def,do,else,extends,
    false,final,finally,for,forSome,if,implicit,import,lazy,
    match,new,null,object,override,package,private,protected,
    return,sealed,super,this,throw,trait,true,try,type,
    val,var,while,with,yield},
  otherkeywords={=>,<-,<\%,<:,>:,\#,@},
  sensitive=true,
  morecomment=[l]{//},
  morecomment=[n]{/*}{*/},
  morestring=[b]",
  morestring=[b]',
  morestring=[b]"""
}[keywords,comments,strings]

% configuration of lstlisting
\lstset{%
	xleftmargin=0.5cm, % expected by LNI
    captionpos=b,      % expected by LNI
    fontadjust=true,
    columns=[c]fixed,
    keepspaces=true,
    tabsize=2,
    basicstyle=\renewcommand{\baselinestretch}{0.95}\ttfamily,
    commentstyle=\itshape,
    keywordstyle=\bfseries,
    mathescape=true,
    escapechar=§,
}

% macro for inline code
\newcommand{\code}[1]{\lstinline[flexiblecolumns=true,basicstyle=\renewcommand{\baselinestretch}{0.95}\ttfamily]{#1}}

%% End: lstlistings

%% Begin: Acronyms
\usepackage[acronym]{glossaries}
\glsdisablehyper

% define acronyms here:
\newacronym{jvm}{JVM}{Java Virtual Machine}
\newacronym{rdbms}{RDBMS}{relational database management system}
\newacronym{orm}{ORM}{object-relational mapping}
\newacronym{oop}{OOP}{Object-oriented Programming}

% define special names here (we do not create a glossary, so no descriptions are required)
\newglossaryentry{dactor}{name={Dactor},plural={Dactors},description={}}
\newglossaryentry{functor}{name={Functor},plural={Functors},description={}}
\newglossaryentry{relation}{name={relation},plural={relations},description={}}
%% End: Acronyms


%% Begin: Macros
\newcommand{\reactdb}[1]{\textsc{ReactDB}}
%% End: Macros


%% correct bad hyphenation here
\hyphenation{net-works semi-conduc-tor}


% Start of page count 
% ----------------------- filled out by publisher/editor
\startpage{1}
\editor{Vorname Nachname et al.}
\booktitle{Konferenztitel}
% -----------------------

\author[Frederic Schneider \and Sebastian Schmidl]{%
Frederic Schneider \and
Sebastian Schmidl\footnote{Hasso-Plattner-Institut, University of Potsdam, Prof.-Dr.-Helmert-Str. 2-3, 14482 Potsdam, \email{{frederic.schneider,sebastian.schmidl}@student.hpi.de}}
}
\title[An Actor Database System for Akka]{An Actor Database System for Akka}

\begin{document}
\maketitle

% Set to number of authors!
% Authors use each one footnote counter, set this to align the remaining ones.
% E.g. 2 authors --> set to 2, next footnote will be 3
\setcounter{footnote}{2}

\begin{abstract}
  System architectures for data-centric applications is commonly comprised of two tiers:
  An application logic tier containing the bulk of the business logic, as well as a data tier, which stores the application's data.
  The fact that these tiers do not typically share a format or representation for data and state information and the resulting mapping issues are commonly referred to as \textit{object-relational impedance mismatch}. % Maybe ORM -> less transparency?

  We build on the concept of an actor database system as proposed by \citet{Shah:reactdb}, which uses actors, as found in the Actor programming model, for data storage.
  Actors are strongly encapsuled objects comprised of state and behavior, which is used to execute concurrent computations.
  The actor database system utilizes the actors as a logically distributed runtime for data storage, as well as query execution.

  We implement a proof-of-concept actor database application framework which enables the definition of application logic as part of shared data storage and application runtime to eliminate the aforementioned impedance mismatch.
  We implement domain actors which provide a typesafe, sql-like data interface for internal behavior definition,
  and present the Functor concept to enable queries and functionality spanning data contained in multiple actor instances.

  We demonstrate the feasibility of the approach, which entails partitioning application data into a large number of actor instances, by evaluating the introduced memory overhead.
  We also discuss challenges that arise from the distributed database concept and how the actor model lends itself to implement solutions for data partitioning and failure handling for distributed, concurrent functionality.
\end{abstract}

\begin{keywords}
Actor Model \and Actor Database System \and Akka \and Database \and Distributed Computing \and Parallelization
\end{keywords}

% Contents
% --------
\input{paper_toc}
% --------

% bibliography
\printbibliography

\end{document}

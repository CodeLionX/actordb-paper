% !TeX root = ../paper.tex
% !TeX encoding = UTF-8
% !TeX spellcheck = en_US

\section{Related Work}

  \subsection{Actor Model}
  % https://www.brianstorti.com/the-actor-model/
  % https://en.wikipedia.org/wiki/Actor_model
  % https://doc.akka.io/docs/akka/2.5.3/scala/guide/actors-intro.html
  % https://link.springer.com/content/pdf/10.1007%2F978-3-642-39038-8.pdf#page=318 (page 302)
  Actor Programming denotes a conceptual model for concurrent computation.
  The core primitive in this model are actors, which are objects comprised of state and behaviour that execute computational tasks independently and concurrently.
  Actor communication takes place exclusively via asynchronous message passing.
  Incoming messages are stored in a mailbox and this allows actors to process messages one at a time.
  An actor can only access and modify its own internal state.
  For affecting other actors, it must rely on exchanging messages.
  On receiving a message an actor can modify its state, create new actors, and send more messages.
  % \enquote{The actor model brings OOP back to the system level with actors appearing to developers very much like the familiar model of interacting objects}~\cite{bernstein:orleans}.

  Various programming languages and libraries implement the actor model and provide the corresponding functionality:
  The Erlang programming language supports concurrency using the actor programming model and is designed for distributed and high-available systems in the telecommunication sector~\cite{armstrong:erlang}.
  Akka implements the actor model for the \gls{jvm} and can be used in the programming languages Java and Scala~\cite{akka}.
  Akka.Net is a community-driven port of the Akka toolkit for the .NET platform in the languages C\# and F\#~\cite{akka.net}.

  \subsection{Virtual Actor Model}
  While the actor programming model allows for low-level interaction with the actor lifecycle management, handling race conditions, physically distributing actors and failure handling, the virtual actor programming model raises the abstraction level of actor objects.
  It still adheres to the plain actor model, but treats actors as virtual entities.
  In this model actor lifecycle management and distribution is managed by the framework or runtime and not by the application developer.
  The physical location of the actor is transparent to the application.
  This model is comparable to virtual memory, which allows processes to access a memory address whether it is currently physically available or not.

  The virtual actor model was introduced with the Orleans framework for .NET by Microsoft Research~\cite{bernstein:orleans}.
  In Orleans, actors (virtually) exist at any time.
  There is no possibility to create an actor explicitly.
  Instead this task is handled by the runtime and it instantiates actors on demand.
  The runtime resource management deals with unused actors and reclaims their space automatically.
  The Orleans runtime guarantees that actors are always available.
  This means, that in the view of an application developer an actor can never fail.
  The runtime will deal with crashing actors and servers and recreate missing actor incarnations accordingly.
  Stateless actors have multiple incarnations to be able to scale out hot actors.

  The corresponding framework for the \gls{jvm} is Orbit~\cite{orbit}.
  In Orbit, a virtual actor can be active or inactive.
  But those two states are transparent to the application developer.
  Similar to Orleans, messages to an inactive actor in Orbit will activate it, so the actor can receive the sent message.
  Actor state in Orbit is usually stored in a database and this allows activation and deactivation of actors by the runtime.
  During activation of an actor, the state is read from database and the actor gets active.
  Before the actor is deactivated, its state is written back to the database.

  %Discussion about virtual actors in Akka.NET (also applies to Akka): \url{https://github.com/akkadotnet/akka.net/issues/756}
  %-- somewhat implemented with cluster sharding project
  Akka provides similar functionality with Akka Cluster Sharding~\cite{akka:clustersharding}.
  It introduces the separation of virtual and physical Akka actors.
  Actors are first grouped into shards, which can then be located at different physical locations.
  The distribution of the shards and their actors is managed by the framework and it also deals with the routing of the messages and re-balancing the shards to different nodes based on different strategies.
  A developer must not know, where a specific actor is physically located to send messages to it.
  Akka Cluster Sharding also supports passivation of actors.
  This means that unused persistent actors can be unloaded to free up memory (passivation).
  A message to passive actors will instruct the framework to load it back into memory.

  In this paper we denote virtual actors as vactors, because this clearly distinguishes actors of the actor model (actor) from those of the virtual actor model (vactor).


  \subsection{Actor Database Systems}
  \begin{itemize}
    \item manifesto~\cite{manifesto}
      \begin{itemize}
        \item actors and virtual actors: state management in application (main mem. or persistent)
        \item application also has to handle faults and consistency requirements
        \item no atomicity or consistency guarantees out-of-the-box
        \item need of state management guarantees in actor systems --> actor database system
        \item give vision about new database system paradigm: actor database systems
        \item list required features of such a system (tenants)
      \end{itemize}
    \item --> existing work in Orleans
    \item ReactDB (same author than manifesto)~\cite{Shah:reactdb}
    \item actorDB~\cite{biokoda:actordb}
      \begin{itemize}
        \item ACID-conform
        \item \enquote{An actor is a mini fully relational SQL database.}~\cite{biokoda:actordb}
        \item distributed consensus algorithm (Raft) with leaders
        \item relies on SQLite
        \item explicit partitioning and replication
        \item uses mySQL-protocol
      \end{itemize}
  \end{itemize}

% Possibly last thing in the Introduction before overview of the rest of the paper
\begin{enumerate}
  \item dass halt meistens was in Orleans probiert wurde und bisher nicht in Akka
\end{enumerate}
% !TeX root = ../paper.tex
% !TeX encoding = UTF-8
% !TeX spellcheck = en_US

\section{Domain Actor Database Framework}\label{sec:framework}
% Describe implementation details and framework design

After describing the general concept of an \textit{Actor Database System}, we present the implementation of a proof-of-concept application development framework, which allows for the definition of an application's data model and the provisioning of corresponding \glspl{dactor} as part of the application runtime.

The framework provides the aforementioned features for developers:
\begin{itemize}
  \item Declarative data model definition using \glspl{dactor}
  \item SQL-like interface for accessing and manipulating internal data within \glspl{dactor}
  \item \Gls{functor} objects for procedures, which rely on data contained in multiple \glspl{dactor}
\end{itemize}

% Tenets
These features relate to three of the four core tenets that define an \textit{Actor Database System} according to \citeauthor{manifesto}'s manifesto~\cite{manifesto}:
\textbf{Tenet 1} calls for the use of actors to achieve a modular logical model for data encapsulation.
\Gls{dactor} instances are in-memory storekeepers for application data.
They satisfy the actor definition and support high modularity.
Furthermore, \glspl{dactor} provide a model for concurrent computation of predefined functionality that increases locality of data accesses.
This is part of the requirements described in \textbf{Tenet 2}.
All communication between \glspl{dactor} is asynchronous to leverage the advantages of increasingly parallel hardware.
\Glspl{functor} allow for the definition of functionality using multiple actors' data and are executed asynchronously as well.
The possibility to define these \glspl{functor} as well as \gls{dactor} behavior in a declarative way relates to the second part of \textbf{Tenet 3}.
Due to their single-threaded computation model, \glspl{dactor} can be argued to enforce internal consistency by default.
In principle, \glspl{functor} allow for the implementation of further transaction protocols to ensure inter-\gls{dactor} consistency guarantees.
Since the requirements listed under \textbf{Tenet 4} relate mostly to production-ready database products, we do not implement or investigate them in this research project.
The functionality described by \citeauthor{manifesto} in tenets one through three suffices the investigation of the practical feasibility and implications of the actor database concept.

% Tech
Our proof-of-concept framework is developed using the Scala programming language and the Akka framework.
While the Orleans framework, which has been used in earlier related work, provides virtual actors with automatic lifecycle management, Akka allows for and requires direct control over an actor's lifecycle.
% As \glspl{dactor} directly build on the provided Akka actor implementation, these considerations directly apply to them.
These considerations directly apply to \glspl{dactor} as they are built on top of the Akka actor implementation.
% More Discussion of the implication that this allows for more detailed configuration and custom \gls{dactor} idle behavior in Discussion

The direct control of the \glspl{dactor} lifecycles, in principle, enables close configuration of the in-memory state of the database system.
Frequently accessed \glspl{dactor} can be kept in memory while ones, which haven't been involved in recent computational effort, could be temporarily persisted to disk.
The virtual actor system implemented in the Orleans framework withdraws such control, making custom, database specific logic for configuring \gls{dactor} availability impossible.
This makes our prototype on Akka different from the related implementations in Orleans.

\subsection{Domain Actor Design}\label{subsec:domain_actor_design}

To illustrate the definition of \code{Dactor}s, we use the example introduced in \cref{sec:dactors}, which is schematically depicted in \cref{fig:dactor_diagram}.
The corresponding definition of the \code{Film}'s data model is shown in \cref{lst:film_definition}.
\begin{lstlisting}[float, caption={Film Dactor type definition using the presented framework.}, label={lst:film_definition}, language=Scala]
object Film {
  // Definitions of relations contained in Film type Dactors
  object Info extends RelationDef {
    // column with name "title" and data-type String:
    val title = ColumnDef[String]("title")
    val description = ColumnDef[String]("description")
    // rich datetime data-type column:
    val release = ColumnDef[ZonedDateTime]("release")
  }

  object Cast extends RelationDef {
    // foreign key relating to an Actor type Dactor instance:
    val actorId = ColumnDef[Id]("actor_id")
    val name = ColumnDef[String]("actor_name")
    val rolename = ColumnDef[String]("role_name")
  }
} 

// Dactor type definition.
class Film(id: Id) extends Dactor(id) {
  override protected val relations = {
    // special relation type ensuring domain-logic constraint
    Film.Info -> SingleRowRelation(Film.Info),
    Film.Cast -> RowRelation(Film.Cast)
  }
  override def receive: Receive = ??? // Dactor behavior
}
\end{lstlisting}

Developers can model the application's domain objects by defining \gls{dactor} types as subclasses of the framework-provided \code{Dactor} class.
Instances of such user-defined \gls{dactor} types are managed by the framework and are available for messaging in a consistent namespace.
A \gls{dactor} type definition is comprised of the type's internal data layout, i.e. \glspl{relation}, and its functionality to process said data.
In detail, a \gls{dactor} type's data model definition consists of \glspl{relation} with unique schemas.
A \gls{relation}'s schema is comprised of column definitions, whereby every column is defined by a name and a data type.
Using the column's predefined data types, all functions support compile-time type-safety.
Due to the \gls{dactor} system sharing the application's runtime and programming environment, these data or object types are equal to the types handled in any application logic.
Thus, this approach helps eliminate the impedance mismatch between application logic and data tier with regard to handled data types and object (de-)serialization.
Note that, since \glspl{dactor} can logically encapsulate multiple data sets and state information, they can and often do comprise multiple \glspl{relation}.

The partial denormalization of information, which relates to \code{Actor} instances, into the \code{Film.Cast} \gls{relation} of \code{Film} type \glspl{dactor},
enables the computation of and response to common queries, such as requesting info on a film and their cast list, from within a single \gls{dactor} instance.
The trade-off for this is that inserting new information about an actor (person) starring in a film becomes an operation that involves a multitude of \gls{dactor} instances.
Implementation examples for this functionality and the trade-off are discussed in the following Section~\ref{subsec:multi_dactor_queries}.

\subsection[Multi-Dactor Queries]{Multi-\Gls{dactor} Queries}\label{subsec:multi_dactor_queries}

Explicit messaging and complementary actor behavior are used to implement the cascading communication pattern described in Section~\ref{sec:concept}.
Besides this, \glspl{functor} are the framework's system, which enable inter-\gls{dactor} communication and computations.
While the basic functionality of the two \gls{functor} types for sequential and concurrent computation is discussed in Section~\ref{sec:concept}, this section focuses on the definition of a concrete function integrating data from multiple \glspl{dactor}.

Functors are actors with a limited lifetime.
They are instantiated once a defined Functor is called by a Dactor or other actor and deleted once they return the final response or encounter a failure.
Functors are always called from an Akka actor as a child actor.
This Akka-specific hierarchical relationship enables notifying the calling actor even in case of an unforseen crash of the Functor themselves which in turn allows to trigger error handling, e.g. retrying the Functor execution.

Returning to the web application example, we can define simple functionality concerning instances of both \gls{dactor} types defined in Section~\ref{subsec:domain_actor_design}.
Adding a new character role to a film, played by a specific actor, involves changes to \code{Film} and \code{Actor} instances.
We describe the implementation of this functionality using the concurrent and sequential computation patterns, introduced in \cref{sec:dactor_comm}.

\Cref{fig:functor_diagram} displays the message-flow between the involved \glspl{dactor} and Functors, as well as their supervision relations.
To provide a top-level function that takes only an \code{Actor} id, a \code{Film} id, as well as a rolename and adds the corresponding Actor information to the Film's cast list and vice versa, this example utilizes a concurrent Functor wrapping two sequential Functors.
Both sequential Functors are comprised of two subsequent steps:
The first sequential Functor queries information from the Film Dactor instance, which is needed to add the film to the Actor Dactor's filmography \gls{relation}, i.e. the film's title.
Then it sends a message to the Actor Dactor containing said data to trigger the insertion of the film.
The second sequential Functor queries the Actor instance's data, which is necessary to add them to the Film's cast \gls{relation}, i.e. their name.
Using this data and their rolename the Functor then sends another message to the Film Dactor to trigger an insertion into its cast \gls{relation}.
Both sequential Functors are executed in parallel using a concurrent Functor, which only sends a successful response to its caller after both sequential Functors have sent their responses to the concurrent Functor.

\begin{figure}[t]
  \centering
  \includestandalone[width=\textwidth]{pictures/tikz/functor_diagram}
  \caption{Component diagram indicating the message flow through Functor objects and their supervision by the calling actor or \gls{dactor}. Arrow and dashed arrow pairs indicate corresponding request and response messages. The outgoing requests of each sequential functor are numbered to indicate their order.}
  \label{fig:functor_diagram}
\end{figure}

Using the parent-child relationship between calling actors and Functors enables a framework-level supervision of Functors, which are currently computing or waiting on results from down-stream \glspl{dactor}.
The supervision concept between parent and child actors in the Akka framework, termed parental supervision, enables passing back the control flow to the parent actor in case that a child actor, in this case a Functor, fails.
The calling actor can then react to the failure by resuming or restarting the Functor, stopping this computation, or escalating the failure to its own parent.
The correct course of action depends on the desired application behavior.

\subsection[Framework Discussion]{Framework Discussion}\label{subsec:framework_discussion}

We presented an actor database framework and showcased some of its functionality in detail.
In particular, the described feature set is aimed to alleviate some of the problems, which arise from a distributed database system.
In this section we discuss three core problems, which are specific or pose additional challenges in a distributed setting:

\textbf{Data partitioning} in an actor database system differs fundamentally from common partitioning techniques used in relational databases.
While large tables are typically partitioned based on a specific column's value or the hash thereof, our framework provides \glspl{dactor} as entities for data encapsulation and partitioning.
\Glspl{dactor} can be provisioned across multiple virtual runtimes and physical machines, because every \gls{dactor} instance is independent of the others and the only mean of communication is message passing.
As such, they provide flexible, fine-grained data partitioning based on application needs.

From the distributed setting of the database system the new problem of partition or \textbf{actor discovery} arises.
The framework maintains a unified namespace, in which each \gls{dactor} instance is identified by its \gls{dactor} type and a unique id.
In fact, querying a specific \gls{dactor} just requires obtaining the messaging address from the nameservice and sending a message to it as shown in \cref{lst:dactor_lookup}.
In the case of a multi-node deployment, this is complemented by the functionality of Akka Cluster Sharding~\cite{akka:clustersharding}, which routes the messages to the right physical host.
\begin{lstlisting}[float,caption={Lookup of and querying from a Film dactor instance.}, label={lst:dactor_lookup}, language=Scala]
val ref = Dactor.dactorSelection(classOf[Film], filmId)
ref ! SelectAllFromRelation.Request("film_info")
\end{lstlisting}

Finally, \textbf{failure handling}, especially with regard to computations relying on multiple \glspl{dactor}' data, requires careful monitoring due to \gls{dactor} distribution.
Building on the Akka framework's parent-child supervision concept, our framework allows for transparent failure handling configurations.
Failures can be handled within \glspl{dactor} if appropriate.
In case of multi-\gls{dactor} queries a fail-fast approach is chosen to allow calling actors or \glspl{dactor} to react to exceptions in a timely manner.

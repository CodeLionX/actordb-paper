% !TeX root = ../paper.tex
% !TeX encoding = UTF-8
% !TeX spellcheck = en_US

\section{Domain Actor Database Framework}\label{sec:framework}
% Describe implementation details and framework design

After describing the general concept of an \textit{Actor Database System}, we present the implementation of an proof-of-concept application development framework, which allows for the definition of an application's data model and the provisioning of corresponding \glspl{dactor} as part of the application runtime.

The framework provides the aforementioned features for developers:
\begin{itemize}
  \item Declarative data model definition using \glspl{dactor}
  \item SQL-like interface for accessing and manipulating internal data within \glspl{dactor}
  \item \Gls{functor} objects for procedures, which rely on data contained in multiple \glspl{dactor}
\end{itemize}

% Tenets
These features relate to three of the four core tenets that define an \textit{Actor Database System} according to \citeauthor{manifesto}'s manifesto~\cite{manifesto}:
\textbf{Tenet 1} calls for the use of actors to achieve a modular logical model for data encapsulation.
\Gls{dactor} instances are in-memory storekeepers for application data.
They satisfy the actor definition and support high modularity.
Furthermore, \glspl{dactor} provide a model for concurrent computation of predefined functionality that increases locality of data accesses.
This is part of the requirements described in \textbf{Tenet 2}.
All communication between \glspl{dactor} is asynchronous to leverage the advantages of increasingly parallel hardware.
\Glspl{functor} allow for the definition of functionality using multiple actors' data and are executed asynchronously as well.
The possibility to define these \glspl{functor} as well as \gls{dactor} behavior in a declarative way relates to the second part of \textbf{Tenet 3}.
Due to their single-threaded computation model, \glspl{dactor} can be argued to enforce internal consistency by default.
In principle, \glspl{functor} allow for the implementation of further transaction protocols to ensure inter-\gls{dactor} consistency guarantees.
Since the requirements listed under \textbf{Tenet 4} relate mostly to production-ready database products, we do not implement or investigate them in this research project.
The functionality described by \citeauthor{manifesto} in tenets one through three suffices the investigation of the practical feasibility and implications of the actor database concept.

% Tech
Our proof-of-concept framework is developed using the Scala programming language and the Akka framework.
This implementation marks, to the best of our knowledge, the first practical examination of the actor database concept using Akka and the classical actor programming model.
While the Orleans framework, which has been used in earlier related work, provides virtual actors with automatic lifecycle management, Akka allows for and requires direct control over an actor's lifecycle.
% As \glspl{dactor} directly build on the provided Akka actor implementation, these considerations directly apply to them.
These considerations directly apply to \glspl{dactor} as they are built on top of the Akka actor implementation.
% More Discussion of the implication that this allows for more detailed configuration and custom \gls{dactor} idle behavior in Discussion

The direct control of the \glspl{dactor} lifecycles, in principle, enables close configuration of the in-memory state of the database system.
Frequently accessed \glspl{dactor} can be kept in memory while ones, which haven't been involved in recent computational effort, could be temporarily persisted to disk.
The virtual actor system implemented in the Orleans framework withdraws such control, making custom, database specific logic for configuring \gls{dactor} availability impossible.

\subsection{Internal Data Management}

In-memory data that is contained within a \gls{dactor} instance is managed in a data structure called \gls{relation}.
% Relations contain a multiset of records or tuples which follow a pre-defined schema.
A \gls{relation}, similarly to a table in the relational database model, is defined as a multiset of tuples following a predefined schema.
This schema is comprised of column definitions whereby every column is defined by a name and a data type.
%They provide a type-safe, SQL-like interface for querying and manipulating contained data.
%The features and interface they provide is inspired by SQL to allow for simple data manipulation using known and proven syntax and semantics when defining a \gls{dactor}'s behavior.
%All manipulation and query functionality on a relation's data provide compile-time type safety based on these column's predefined data types.
\Glspl{relation} provide an SQL-like interface to query and manipulate the contained data, so known and proven syntax and semantics can be used to define \gls{dactor}-behavior.
Using the column's predefined data types, all functions support compile-time type-safety.
Due to the \gls{dactor} system sharing the application's runtime and programming environment, these data or object types are equal to the types handled in any application logic.
Thus, this approach eliminates the impedance mismatch between application logic and data tier with regard to handled data types and object (de-)serialization.

% This could be Usage / Discussion
Due to the similarities between \glspl{relation} and tables known from the relational database model, we want to discuss the conceptual differences between the two.
While \glspl{relation} can be used to model much of the same information as is commonly represented using tables, the \textit{Actor Database System} and specifically the concept and framework we present relies on \glspl{dactor} as models for application domain objects.

We consider the following example of a web application with information on films and (human) actors similar to the imdb.com or rottentomatoes.com websites.
A traditional data layout might be comprised of two tables containing information on films and actors, respectively.
Using \glspl{dactor} to represent discrete domain-level objects, a single \gls{dactor} of the type film represents a single, specific film and encapsulates all corresponding data.
This approach to layout an application's data results in much smaller \gls{relation} sizes than typical table sizes and enables many business-logic-driven approaches to scaling, data partitioning, and caching.
The tradeoff to take into account, however, is a large number of \gls{dactor} instances encapsulating the application's data.

\subsection{Domain Actor Design}\label{subsec:domain_actor_design}

\begin{figure}[t]
  \centering
  \includestandalone[width=\textwidth]{pictures/tikz/dactor_diagram}
  \caption{Class diagram of the \texttt{Film} and \texttt{Actor} Dactor types defined in the example application. Both Dactor types contain two \glspl{relation}. Dactor id attributes in the relations are italised and realise a many-to-many relationship between the two Dactor types.}
  \label{fig:dactor_diagram}
\end{figure}

Developers can model the application's domain objects via extending the framework's abstract \code{Dactor} class.
Instances of such user-defined \code{Dactor} types are managed by the framework and are available for messaging in a consistent namespace.
A \code{Dactor} type definition is comprised of the types internal data schema, i.e. \glspl{relation}, and its functionality to process said data.
Note that, since \glspl{dactor} can logically encapsulate multiple data sets and state information, they can and often do comprise multiple \glspl{relation}.

To illustrate the definition of \code{Dactor}s we use the formerly introduced example.
The application in its simplest form consists of two \code{Dactor} types modeling the domain entities: \code{Film} and \code{Actor}.

Both, \code{Film} and \code{Actor} \gls{dactor} instances contain two \glspl{relation} and store information about a specific film or actor, respectively.
In addition, \code{Actor} instances keep some information on related \code{Film}s, i.e. those the \code{Actor} has starred in, including the \code{Film}'s \gls{dactor} id, and vice-versa.
This realises a many-to-many relationship between both \gls{dactor} types.
The data model is detailed in Figure~\ref{fig:dactor_diagram}.
The definition of the \code{Film}'s data model is shown in Listing~\ref{lst:film_definition}.

\begin{minipage}{1.0\linewidth}
\begin{lstlisting}[caption=Film Dactor type definition using the actordb framework, label=lst:film_definition, language=Scala]
object Film {
  // Definitions of relations contained in Film type Dactors
  object Info extends RelationDef {
    // column with name "title" and data-type String:
    val title = ColumnDef[String]("title")
    val description = ColumnDef[String]("description")
    // rich datetime data-type column:
    val release = ColumnDef[ZonedDateTime]("release")
  }

  object Cast extends RelationDef {
    // foreign key relating to an Actor type Dactor instance:
    val actorId = ColumnDef[Id]("actor_id")
    val name = ColumnDef[String]("actor_name")
    val rolename = ColumnDef[String]("role_name")
  }
} 

// Dactor type definition.
class Film(id: Id) extends Dactor(id) {
  override protected val relations = {
    // special relation type ensuring domain-logic constraint
    Film.Info -> SingleRowRelation(Film.Info),
    Film.Cast -> RowRelation(Film.Cast)
  }
  override def receive: Receive = ??? // Dactor behavior
}
\end{lstlisting}
\end{minipage}
% \begin{lstlisting}[caption=Actor \code{Dactor} type definition using the actordb framework, label=lst:actor_definition, language=Scala]
%
% \end{lstlisting}

Note the partial denormalization of data on the respective domain objects into two \glspl{dactor}.
Some information, which relates directly to \code{Actor} instances is replicated in the \code{Film.Cast} \gls{relation} of \code{Film} type \glspl{dactor}.
This enables the computation and response to common queries, such as requesting info on a film and their cast list, from a single \gls{dactor} instance.
More complex queries, e.g. finding out, which films star the same actors as a given film, involve data and computation on multiple \gls{dactor} instances of both \gls{dactor} types.
The trade-off for this is that inserting new information about an actor (person) starring in a film becomes a functionality involving two \gls{dactor} instances and types.
This trade-off and implementation examples for this functionality are discussed in the following Section~\ref{subsec:multi_dactor_queries}.


\subsection[Multi-Dactor Queries]{Multi-\Gls{dactor} Queries}\label{subsec:multi_dactor_queries}

Explicit messaging and complementary actor behavior are used to implement the cascading communication pattern described in Section~\ref{sec:concept}.
Besides this, \glspl{functor} are the framework's system, which enable inter-\gls{dactor} communication and computations.
While the basic functionality of the two \gls{functor} types for sequential and concurrent computation is discussed in Section~\ref{sec:concept}, this section focuses on the definition of a concrete function integrating data from multiple \glspl{dactor}.
We also discuss the implications of using the respective messaging patterns corresponding to these types of \glspl{functor} on computational scalability.

Functors are actors with a limited lifetime.
They are instantiated once a defined Functor is called by a Dactor or other actor and deleted once they return the final response or encounter a failure.
Functors are always called from an Akka actor as a child actor.
This Akka-specific hierarchical relationship enables notifying the calling actor even in case of an unforseen crash of the Functor themselves which in turn allows to trigger error handling, e.g. retrying the Functor execution.

Returning to the web application example, we can define simple functionality concerning instances of both \gls{dactor} types defined in Section~\ref{subsec:domain_actor_design}.
Adding a new character role to a film, played by a specific actor involves changes to \code{Film} and \code{Actor} instances.
We describe the implementation of this functionality using all three of the introduced communication patterns.

To provide a top-level function call that takes only an Actor instance's id, a Film instance's id, as well a rolename and adds the corresponding Actor information to the identified Film instance's cast list and vice versa, this example utilizes a concurrent Functor wrapping two sequential Functors.
Both sequential Functors are comprised of two subsequent steps:
The first one queries information from the Film Dactor instance which is needed to add the film to the Actor Dactor's filmography \gls{relation}, i.e. the films title.
Then it sends a message to the Actor Dactor containing said data to trigger the insertion of the film.
The second sequential Functor queries the Actor instance's data which is necessary to add them to the Film's cast \gls{relation}, i.e. their name. Using this data and their rolename the Functor then sends another message to the Film Dactor to trigger an insert into it's cast \gls{relation}.
Both sequential Functors are executed in parallel using a concurrent Functor which only sends a success-response message to it's caller after both sequential Functor's have sent their responses to the concurrent Functor.
Using the parent-child relationship between calling actors and Functors enables a framework-level supervision of ongoing Functors.
Figure \ref{fig:functor_example} displays the message-flow between the involved Dactors and Functors, as well as their supervision \glspl{relation}.

% TODO figure

% Discussion


% !TeX root = ../paper.tex
% !TeX encoding = UTF-8
% !TeX spellcheck = en_US

\section{Domain Actor Database Framework}\label{sec:framework}
% Describe implementation details and framework design

After describing the general concept of an \textit{Actor Database System}, we present the implementation of an application development framework which allows for the definition of an application's data model and the provisioning of corresponding Dactors as part of the application runtime.

The goal for this research implementation is a proof-of-concept framework which allows developers to declaratively define their application's data model
and which automatically instantiates special-type actors, i.e. Dactors, corresponding to defined domain-object entities.
These Dactors encapsulate the respective application data and provide a SQL-like interface for data access and manipulation internally.
This interface is used to define actor behaviour and functionality.
Functionality that relies on data contained within multiple Dactor objects is defined explicitely using Functor objects which have been described conceptually in the former section \ref{sec:concept}.
These Functors allow to define inter-Dactor functionality without breaking the strictly asynchronous communication between actors or halting an actors single-threaded computation process.

% Tenets
This framework's feature set relates to three of the four core tenets that define an \textit{Actor Database System} according to \citeauthor{manifesto}'s manifesto \cite{manifesto}:
\textbf{Tenet 1} calls for the use of actors to achieve a modular logical model for data encapsulation.
Dactor instances are in-memory storekeepers for an application's data and provide an actor construct which encapsulates data.
Furthermore, Dactors provide a model for concurrent computation of predefined functionality that processes a Dactor's data.
This is part of the requirements described in \textbf{Tenet 2}.
Functors allow for the definition of functionality using multiple actors's data and are excuted asynchronously.
The possibility to predefine these Functors as well as Dactor behaviour relates to the second part of \textbf{Tenet 3}.
Due to their single-threaded computation model, Dactors can be argued to enforce internal consistency by default.
In principle, Functors allow for the implementation of further transaction protocols to ensure cross-Dactor consistency guarantees.
Since the requirements listed under \textbf{Tenet 4} relate mostly to production-ready database products, we do not implement or investigate them in this research project.
The functionality described by \citeauthor{manifesto} in tenets one through three allows for investigation of the practical feasibility and implications of the actor database concept.

% Tech
The proof-of-concept framework has been developed using the Scala programming language and the Akka framework for actor programming.
This implementation marks, to the best of our knowledge, the first practical examination of the actor database concept using Akka and the classical actor concept.
While the Orleans framework, having been used in earlier related work, provides virtual actors with automatic lifecycle management, Akka allows for and requires direct control over an actors lifecycle.
% As Dactors directly build on the provided Akka actor implementation, these considerations directly apply to them.
These considerations directly apply to Dactors as they are built on top of the Akka actor implementation.
% More Discussion of the implication that this allows for more detailed configuration and custom Dactor idle behaviour in Discussion

\subsection{Internal Data Management}

In-memory data that is contained within a Dactor instance is managed in a data structure called relation.
Relations contain a multiset of records or tuples which follow a pre-defined schema.
They provide a type-safe, SQL-like interface for querying and manipulating contained data.
The features and interface they provide is inspired by SQL to allow for simple data manipulation using known and proven syntax and semantics when defining a Dactors behavior.

A relation, similarly to a table in the relational database model, is defined as a multiset of tuples following a predefined schema.
This schema is comprised of column definitions whereby every column is defined by a name and data type.
All manipulation and query functionality on a relation's data provide compile-time type safety based on these column's predefined data types.
Due to the Dactor system sharing the application's runtime and programming environment, these data or object types are equal to the types handled in any application logic.
Thus, this approach eliminates the impedance mismatch between application logic and data tier with regard to handled data types and object (de-)serialization.

% This could be Usage / Discussion
Due to the similarities between relations and tables known from the relational database model, we want to discuss the conceptual differences between the two.
While relations can be used to model much of the same information as is commonly represented using tables, the \textit{Actor Database System} and specifically the concept and framework we present relies on Dactors as models for application domain objects.

For clarification consider the following example of a web application with information on films and (real-life) actors similar to the imdb.com or rottentomatoes.com websites, a traditional data layout might be comprised of two tables containing information on films and actors, respectively.
Using Dactors to represent discrete domain-level objects, a single Dactor of the type film represents a single, specific film and encapsulates all corresponding data.
This approach to layouting an application's data results in much smaller relation sizes than typical table sizes and enables many business-logic--driven approaches to scaling, data partitioning, and caching.

\subsection{Domain Actor Design}\label{subsec:domain_actor_design}

The presented framework provides the abstract Dactor class which is extended by developers to model their application's domain object classes which encapsulate application data.
Instances of such user-defined Dactor types are managed by the framework and are available for messaging in a consistent namespace.
A Dactor type definition is comprised of the types internal data schema, i.e. relations, and it's functionality to process on said data.
Note that, since Dactors can logically encapsulate multiple data sets and state information, they can and often do comprise multiple relations.

To illustrate, we present the Dactor definition in terms of data model and functionality using the presented framework with regard to the formerly considered example.
This application in it's simplest form comprises two Dactor types modelling the concerns of the domain-specific object types of Film and Actor.
We capitalize these Dactor types to eliminate the ambiguity of the Dactor type Actor and the general actor concept within the actor model.

Film Dactor instances comprise information on a specific stored film, such as its title, release date, a description, and a unique id, but also information about its starring actors:
The actor's (person) name, the name of the role they portray, as well as the corresponding Actor instance's id.
The Actor instances comprise additional information on the person such as the date of birth, but also a list of films they starred in, including the name of the role they portray, the film's title, and the Film's id.
The definition of the Film Dactor's data model is shown in listing \ref{lst:film_definition}.

\begin{lstlisting}[caption=Film Dactor type definition using the actordb framework, label=lst:film_definition, language=Scala]
object Film {
  /**
   * Definitions of relations contained in Film type Dactors.
   */
  object Info extends RelationDef {
    // column with name ``title'' and data-type String:
    val title = ColumnDef[String](``title'')
    val description = ColumnDef[String](``description'')
    // rich datetime data-type column:
    val release = ColumnDef[ZonedDateTime](``release'')
  }

  object Cast extends RelationDef {
    // foreign key relating to an Actor type Dactor instance:
    val actorId = ColumnDef[Id](``actor_id'')
    val name = ColumnDef[String](``actor_name'')
    val rolename = ColumnDef[String](``role_name'')
  }
} 

/**
 * Dactor type definition.
 */
class Film(id: Id) extends Dactor(id) {
  override protected val relations = {
    // special relation type ensuring domain-logic constraint
    Film.Info -> SingleRowRelation(Film.Info),
    Film.Cast -> RowRelation(Film.Cast)
  }
}
\end{lstlisting}

% \begin{lstlisting}[caption=Actor Dactor type definition using the actordb framework, label=lst:actor_definition, language=Scala]
%
% \end{lstlisting}

Note the partial denormalization of data on the respective domain objects into two Dactors.
Some information which relate directly to Actor instances is replicated in the Cast relation of Film type Dactors.
This enables the computation and response to common queries, such as requesting info on a film and their cast list, from a singular Dactor instance.
More complex queries, e.g. finding out which films star the same actors as a given film, involve data and computation on multiple Dactor instances of both Dactor types.
The tradeoff for this is that inserting new information about an actor (person) starring in a film becomes a functionality involving two Dactor instances and types.
This tradeoff and implementation examples for this functionality are discussed in the following section \ref{subsec:multi_dactor_queries}.


\subsection{Multi-Dactor Queries}\label{subsec:multi_dactor_queries}

Explicit messaging and complementary actor behaviors are used to implement the cascading communication pattern described in section \ref{sec:concept}.
Besides this, Functors are the framework's system which enables inter-Dactor communication and functionality.
The basic functionality of the two Functor types provided by the framework, sequential and concurrent, are discussed in section \ref{sec:concept}.
This section focuses on the relationship between Functors and their callers and callees, as well as demonstrating a specific usage example.
We also discuss the implications of using the respective messaging patterns corresponding to these types of Functors on computational scalability.

Functors are actors with a limited lifetime.
They are instantiated once a defined Functor is called by a Dactor or other actor and deleted once they return the final response or encounter a failure.
Functors are always called from an Akka actor as a child actor.
This Akka-specific hierarchical relationship enables notifying the calling actor even in case of an unforseen crash of the Functor themselves which in turn allows to trigger error handling, e.g. retrying the Functor execution.

Returning to the web application example we can define simple functionality concerning instances of both Dactor types discussed in section \ref{subsec:domain_actor_design}.
Adding a new character role to a film, played by a specific actor involves changes to Dactor instances of the types Film and Actor.
We describe the implementation of this functionality using nested sequential and concurrent Functors.

To provide a top-level function call that takes only an Actor instance's id, a Film instance's id, as well a rolename and adds the corresponding Actor information to the identified Film instance's cast list and vice versa, this example utilizes a concurrent Functor wrapping two sequential Functors.
Both sequential Functors are comprised of two subsequent steps:
The first one queries information from the Film Dactor instance which is needed to add the film to the Actor Dactor's filmography relation, i.e. the films title.
Then it sends a message to the Actor Dactor containing said data to trigger the insertion of the film.
The second sequential Functor queries the Actor instance's data which is necessary to add them to the Film's cast relation, i.e. their name. Using this data and their rolename the Functor then sends another message to the Film Dactor to trigger an insert into it's cast relation.
Both sequential Functors are executed in parallel using a concurrent Functor which only sends a success-response message to it's caller after both sequential Functor's have sent their responses to the concurrent Functor.
Using the parent-child relationship between calling actors and Functors enables a framework-level supervision of ongoing Functors.
Figure \ref{fig:functor_example} displays the message-flow between the involved Dactors and Functors, as well as their supervision relations.

% TODO figure

% Discussion



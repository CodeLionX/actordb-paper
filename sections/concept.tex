% !TeX root = ../paper.tex
% !TeX encoding = UTF-8
% !TeX spellcheck = en_US

\section{Domain Actor Database Concept}\label{sec:concept}
  Our concept is based on the idea of actor database systems as introduced by~\citeauthor{manifesto}~\cite{manifesto}.
  Database systems should be a distributed runtime and provide the appropriate programming abstraction.
  This is solved by actor database systems.
  The combination of application and data tier provides considerable advantages for data-centric systems.
  This architecture allows leveraging domain and application knowledge to dynamically model the data layout, especially concerning data partitioning and replication schemes.
  This makes the database system modular, cloud-ready and scalable.
  % add here: why is it better to use domain knowledge to partition and distribute data?

  \subsection{Domain Actors}\label{sec:dactors}
    Similar to \citeauthor{Shah:reactdb}~\cite{Shah:reactdb}, we introduce a special type of actor, called Domain Actor (\gls{dactor} for short), that acts as an application-defined scaling unit.
    \glspl{dactor} can be used to model application-domain objects and encapsulate the object's state and application logic in an actor.
    Using actors for this enforces technical encapsulation of state access due to the purely private state in actors and the need of explicit asynchronous messaging between the actors.
    This makes it easier to reason about state changes, bugs and other failures, as only code within the \gls{dactor} can change the corresponding state.

    As \glspl{dactor} not only contain data, but also the corresponding domain logic, computation is executed concurrently.
    This supports designing a modular and extensible database system and improves scalability.
    The system grows naturally based on the load.
    Take an e-commerce application as an example for that:
    If we have a customer entity for every customer in our system, then we can model each entity as a \gls{dactor}.
    This allows our system to scale linearly with the number of customers using it, because each customer is represented with their own \gls{dactor}, which holds their information and performs computations.
    This also illustrates that such a system tends to be more robust compared to monolithic systems, because if customer A's \gls{dactor} crashes for an unknown reason, all other customer \glspl{dactor} are unaffected.

    Actors provide single-threaded semantics, which makes enforcing constraints on data tied to one domain object stored inside \glspl{dactor} no issue.
    It is implemented via application logic within the \gls{dactor}.
    State querying and modification within \glspl{dactor} is possible in a declarative way, but communication across all kind of actors is explicitly defined via asynchronous messages and how the \glspl{dactor} handle the said messages.

  \subsection{Communication between Domain Actors}
    Not all computation can be done with the information residing in a single \gls{dactor}.
    Communication between \glspl{dactor} and other actors is required.
    As already mentioned, this inter-actor communication is realized via explicit asynchronous message passing.
    This allows application developers to chose the right messaging pattern for their use case.
    The choice is heavily influenced by the data layout used for the \glspl{dactor} and sets the level of parallelization and the load on the network of this computation.

    The following messaging patterns are considered in our concept:
    \begin{enumerate}
      \item\label{enum:comp_pattern_1} \textbf{Cascading Computation} \dots
      \item\label{enum:comp_pattern_2} \textbf{Sequential Computation} \dots
      \item\label{enum:comp_pattern_3} \textbf{Concurrent Computation} \dots
    \end{enumerate}
  
    \begin{figure}
      \centering
      \includestandalone{pictures/tikz/cascading_computation}
      \caption{Cascading Computation}
      \label{fig:comp_pattern_1}
    \end{figure}
  
    \begin{figure}
      \centering
      \includestandalone{pictures/tikz/sequential_computation}
      \caption{Sequential Computation}
      \label{fig:comp_pattern_2}
    \end{figure}
  
    \begin{figure}
      \centering
      \includestandalone{pictures/tikz/concurrent_computation}
      \caption{Concurrent Computation}
      \label{fig:comp_pattern_3}
    \end{figure}

    Introducing functors for \ref{enum:comp_pattern_2} and \ref{enum:comp_pattern_3} \dots
    % cameo pattern: http://blog.simplex.software/the-cameo-pattern


  \subsection{Domain Actor Database System}\label{sec:domain_actor_database}
    A domain actor database system is the combination of \glspl{dactor} and other types of actors, which together offer an interface to other applications or clients.
    The domain actor database system provides a distributed runtime for persisting data and performing computations on it.
    Each \gls{dactor} must incarnate a specific \gls{dactor} type.
    This type is defined by application developers and determines the data schema encapsulated by the \gls{dactor} and the messages it can handle.
    \glspl{dactor} are the only place in the system that actually have persistent state.
    Other actors are stateless or have non-durable state, such as intermediate results or actor references.

    \begin{itemize}
      \item How to develop such a system?
        \begin{itemize}
            \item requirements analysis
            \item queries (CRUD or analytics?)
            \item decompose application domain into relations and entities (=dactors)
            \item keep scaling in mind
            \item define dactor types (including their relations and messages)
            \item define messaging patterns for queries (query types)
            \item define entry-points (APIs, interfaces, services) --> stateless actors
            \item implement messaging patterns (using functors)
        \end{itemize}
      \item Consistency?
      \item Transactional guarantees?
    \end{itemize}

    % akka: (i would put this in 4.)
    % message-passing
    % supervision

    % im framework:
    % beispiel


    % how actors solve the impedance mismatch
    % "passt auf, dass hier eure Idee im Vordergrund steht und ihr nicht auf andere Gebiete ausschweift"
    % Computation parallelization on a domain-level object basis
    % Modularity, Testability
